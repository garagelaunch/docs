% ---
% Capitulo de revisão de literatura
% ---
\chapter{Revisão da Literatura}


Neste tópico fazer um estudo e tirar observações mais a fundo a respeito da neurodiversidade. Relatar quando começou os estudos a respeito das pessoas neurodiversas e entender como foi o papel da sociedade no passado, no que nisso acarretou e as mudanças necessárias que surgiram para chegar nos dias de hoje. 
\section{Influência do termo}
As questões neurodiversas em toda sociedade sempre foi um assunto difícil de falar, principalmente porque no passado era tratado com "modelo de tragédia pessoal" \cite{oliver1990politics}. Dessa maneira, foi crescendo o indiferente e o inquestionável a  respeito das pessoas que sofrem algum tipo de neurodiversidade. Entretanto, com os estudos, os teóricos que estudavam a causa mais fundo diziam que esse modelo social estaria equivocado, pois se deixar que essa percepção percorra nossa sociedade não irá trazer as necessárias responsabilidades sociais que devemos atingir para evitar esses equívocos. 
Sendo assim, essa mudança de responsabilidade para o comportamento com as neurodiversidades promove o empoderamentos do indivíduo e sendo assim ele ganha espaço para ser olhado pela sociedade e respeitado por elas. 

\section{Virada de responsabilidades}
Um dos motivos que nos levou a trazer o tema da neurodiversidade para o nosso projeto final está na maneira que o assunto é tratado na sociedade. Há muitos tabus a serem derrubados. Se, por volta de 1960, ainda se culpavam os pais pela pelos seus filhos terem nascido com alguma neurodiversidade e, portanto, impossibilitava o surgimentos de movimentos que pudessem entender e ajudar os familiares,hoje temos algumas conquistas a serem celebradas como projetos de leis que protejam as pessoas neurodiversas graças aos estudos sobre deficiência nos anos 70 que desconstruiu um modelo de responsabilidade única para uma socialmente construída.\cite{ortega}
Todavia, o preconceito, falta de informação e suporte ainda estão presentes no convívio social e dessa maneira, trazer uma rede social focada para sanar dúvidas e compartilhar informações e serviços torna-se esse assunto vivo diante a sociedade.  

\section{Responsabilidade do Aplicativo}
Apontado no artigo de \cite{rios}, evidencia-se que no Brasil há um embate entre profissionais e pais a respeito do movimento neurodiverso, uma vez que a sociedade brasileira ainda está atrelada aos modelos ultrapassados lá da décadas de 1960. Dessa maneira, o diversaGente tem como objetivo tornar-se um ambiente de discussões e compartilhamento de informações sobre o assunto. 
Assim, o comprometimento na era digital que vivemos junto com a neurodiversidade nos remete ao compartilhamento de mais informações a respeito do assuntos com pessoas em diversos lugares, diversas famílias e culturas, pois em diversos artigos apresentam críticas a forma que os métodos atuais de tratamento são usados, uma vez que muitos tratamentos centram-se na deficiência e não na forma humana que deve ser tratado. \cite{machado} 

