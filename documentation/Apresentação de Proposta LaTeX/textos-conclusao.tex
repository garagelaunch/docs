% ---
% Conclusão (outro exemplo de capítulo sem numeração e presente no sumário)
% Dependendo do trabalho desenvolvido ele pode ter uma Conclusão ou Considerações finais
% Para trabalhos de disciplina utilizar Considerações Finais
% ---
\chapter{Considerações Finais}
% Exemplo de como adicionar linha adicional no sumário
Com a matéria de Projeto Integrado I e II (PI1A5 e PI2A6), foi possível vivenciar, de fato, as etapas e desafios que envolvem desenvolver um software e sua documentação.

A experiência de realizar o projeto envolveu, acima de tudo, dificuldades diversas às quais foram necessárias adaptações, fossem elas individuais ou da equipe inteira. Todas essas dificuldades foram de extremo valor, já que puderam mostrar o quão importante são alguns fatores como: comunicação, sinceridade, atenção e proatividade.

Como maiores pontos de dificuldade pelos quais a equipe passou, pode-se citar o conhecimento sobre o tema escolhido, conhecimento técnico sobre as ferramentas de desenvolvimento, gestão de tempo e informações sobre as requisições de cada entrega e, por fim, a documentação LaTeX. Para todos esses pontos de falha foi crucial que houvessem reuniões, alinhamentos e divisão de tarefas balanceadamente, para que não houvesse prejuízo para nenhuma das partes.

Foi preciso esforço de todos os membros da equipe para se informarem acerca do tema escolhido, a neurodiversidade, para que, dessa forma, fossem evitados equívocos de terminologias utilizadas e falta de fontes de informação sobre como é a rotina das pessoas neurodiversas. Quanto aos outros pontos de dificuldade pontuados, foram realizadas diversas agendas e conversas informais para disseminação de conhecimento entre os membros, para que, o quanto antes, todos estivessem alinhados sobre todos os pontos do projeto.

Ao fim do projeto foi possível perceber a evolução de todos os envolvidos, tanto em âmbitos técnicos quanto em quesitos comportamentais, além de ser possível perceber o quão mais entrosada a equipe estava.

