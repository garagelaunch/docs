% ---
% RESUMOS
% ---

% resumo em português
\setlength{\absparsep}{18pt} % ajusta o espaçamento dos parágrafos do resumo
\begin{resumo}

Com o crescimento exponencial do uso da internet na última década, as redes sociais se mostraram o maior centro de socialização da sociedade moderna, local em que são pautadas uma infinidade de assuntos. Percebendo que, em meio a essa infinidade, existe a necessidade de evidenciação de algumas dessas pautas por vezes negligenciadas frente aos tópicos dominantes nas redes, surge o diversaGente. Esse é um aplicativo de rede social focado na melhoria de vida de pessoas com neurodiversidades e suas famílias a partir da criação de um ambiente em que elas sintam-se confortáveis para trocar experiências. E para tornar esse app possível, a metodologia ágil seguida para organizar e desenvolver do projeto baseia-se Scrum. Além disso, toda a documentação foi construída de acordo com as regras \ac{ABNT} e as tecnologias utilizadas no desenvolvimento integram o ecossistema JavaScript, as quais incluem React Native, Typescript, Node.JS, Nest.JS e Prisma.

 \textbf{Palavras-chaves}: rede social. neurodiversidades. projeto.

\end{resumo}

% resumo em inglês
\begin{resumo}[Abstract]
 \begin{otherlanguage*}{english}

With the exponential growth of internet usage in the last decade, social networks proved to be the biggest center of socialization of modern society, where it can be find a multitude of subjects. Realizing that in the midst of this infinity there is a necessity of highlighting some of these themes that are sometimes neglected in comparison to the mainstream topics on the social media, the diversaGente appears. It is a social network app focused on improving the lives of people with neurodiversities and their family by creating an environment where they feel comfortable to exchange experiences. And to build this app, the agile methodology followed to organize and develop the project is based on Scrum. In addition, the complete documentation complies with the \ac{ABNT} rules and the technologies used in development are part of the JavaScript ecosystem, which includes React Native, Typescript, Node.JS, Nest.JS and Prisma.

   \textbf{Keywords}: social networks. neurodiversities. project.
 \end{otherlanguage*}
\end{resumo}