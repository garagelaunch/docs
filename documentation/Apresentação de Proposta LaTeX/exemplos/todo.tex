\subsection{Organizando pendências}

Durante o desenvolvimento de um trabalho escrito é normal que alguns elementos sejam gerados posteriormente, mas é importante se organizar para não esquecer de fazer os ajustes necessários. Para isso recomendo a utilização do pacote \textbf{todonotes} que oferece diversos recursos para gerar lembretes das pendencias. O manual do \textbf{todonotes} está disponível no \autoref{manual-todonotes}\footnote{observe que existe um erro nesse documento, já que a referencia deveria ser Anexo e aparece como Apêndice,  existe um \textit{bug} no abntex2 ao referenciar anexos, para fazer corretamente veja \url{https://github.com/abntex/abntex2/issues/76} e utilize \mostraComandoLaTeX{refanexo} que está disponível nesse modelo.}.

É possível fazer anotações de pendencias inclusive indicando as pessoas responsáveis por elas, % nao mover o todo pois foi feito no meio do paragrafo exatamente para demonstrar um possível problema de formato
\todo[inline,author=Pessoa1]{fazer revisão das imagens do texto} e para facilitar a visualização criar imagens que funcionam como marcadores para figuras que serão incluídas posteriormente.

Cuidado ao utilizar as anotações \textit{inline} pois o texto ficara quebrado, como no paragrafo anterior.


\begin{figure}[htb]
    \centering
	\caption{\label{fig_todo1}Imagem que ainda não foi gerada}
	\missingfigure[figwidth=10cm]{você está atrasado pois ainda não criou esta figura}
	\fonte{dados do Projeto.}
\end{figure}


