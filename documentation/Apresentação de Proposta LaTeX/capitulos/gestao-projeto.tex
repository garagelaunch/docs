\chapter{Gestão do projeto}

A metodologia de gestão de projetos escolhida foi baseada no Scrum. Primeiramente pelo fato da metodologia prezar a independência de todos da equipe, pela rapidez, agilidade de agregar valor para o cliente e produto final e também pelo desenvolvimento de um projeto no qual haverá necessidade de programar uma rede social e o Scrum atualmente é amplamente utilizada por equipes de desenvolvedores e por ser a que os membros da equipe estão mais habituado a trabalhar.

Como foi decidido pela metodologia ágil Scrum, existe uma “hierarquia” e responsabilidades para que o desenvolvimento seja o mais eficiente possível. Para a nossa equipe há os cargos de Scrum Master, Product Owner e Desenvolvedores.

\begin{itemize}
	\item Scrum Master sendo o responsável para que a estrutura do Scrum seja seguida e realizada. Tira empecilhos de membros da equipe para que o desenvolvimento não seja comprometido, verifica a necessidade de adaptação do Scrum e obter um melhor fluxo de trabalho. O Gustavo Freitas ficou com a função de ser o Scrum Master do time.
	
	\item Tayna França Soza é a Product Owner (PO) da equipe e “representa os interesses de stakeholders (no sentido de agregar valor ao negócio), define funcionalidades do produto e faz a gestão do backlog, priorizando tarefas segundo métodos ágeis, como o Scrum.” (Equipe PM3, 2022)
	
	\item Desenvolvedores são os responsáveis de verificar como que as alterações propostas pelo PO vão ser feitas e se podem ser feitas, integrações, modelagem de banco de dados, programação de código do projeto, definir arquitetura e padronização de commits e documentação do código para que ele tenha uma manutenibilidade alta caso seja necessário manutenções futuras e colocar a aplicação em produção sem nenhum problema futura. Os responsáveis por essas funções são: Gabriel Ruiz, Grazielli Berti, João Bispo, Vinicius Soares e Viviane Queiroz.
	
\end{itemize}

Apesar de cada membro da equipe ter as suas funções muito bem estabelecidas como detalhado no \autoref{responsabilidade-integrantes} todos têm uma contribuição na construção do código da aplicação, sendo assim, mesmo que em menor quantidade, o Product Owner e Scrum Master precisam contribuir tanto na parte gerencial quanto técnica de construção da aplicação. 

\begin{quadro}[thb]
	\centering
	\ABNTEXfontereduzida
	\caption[Responsabilidade Integrantes]{Responsabilidade Integrantes}
	\label{responsabilidade-integrantes}
    \begin{tabular}{|l|c|c|c|c|c|c|c|}
		\hline
		\thead{Funções} & \thead{Gabriel} & \thead{Gustavo} & \thead{Grazielli} & \thead{João} & \thead{Tayna} & \thead{Vinicius} & \thead{Viviane}\\
		\hline 
		Back-end & X & X &  & X &  & X &  &
		\hline
		Banco de Dados &  &  &  & X &  &  & X &
		\hline
		Blog & X & X & X & X & X & X & X &
		\hline
		Documentação & X & X & X & X & X & X & X &
		\hline
		Front-end &  &  & X & X & X &  & X &
		\hline
	\end{tabular}
\fonte{Equipe diversaGente (2022)}
\end{quadro}


As Sprints que definem o que será feito em um certo período de tempo e devido ao curto espaço entre as entregas, apresentações e MVP, foi decidido que elas serão de uma semana (sete  dias). 

Para se ter um planejamento eficaz e não passar do prazo em nenhuma entrega, foi acordado que a cada apresentação haverá o back log que precisa ser zerado a cada entrega, dessa maneira é possível ter um visão do esforço que se precisar colocar nas tarefas, e também para distribuí-las conforme conhecimento específico de cada membro da equipe. 

A gestão de projeto utilizada pela Garage Launcher foi baseada na metologia Scrum, seguiu um padrão de 11 sprints de sete dias cada e pode ser vista no \autoref{quadro-atividades} e foram criadas conforme o planejamento da disciplina. Como para o Scrum não é aconselhável fazer mais de duas horas de planejamento para cada semana de Sprint então para poupar tempo toda terça-feira no começo da aula foi utilizado o planejamento e refinamento da Sprint daquela semana. 



\begin{quadro}[htb]
	\centering
	\ABNTEXfontereduzida
	\caption[Sprints]{Sprints}
	\label{quadro-atividades}
	\begin{tabular}{|p{1.5cm}|p{3.0cm}|p{10.0cm}|}
		\hline
		\thead{Sprints} & \thead{Período}  & \thead{Atividades} \\
		\hline
		Sprint1 & 15 - 21/mar  & Realização da separação das equipes, verificar quais as melhores skills de cada um do grupo. Separação das ideias para o tema do projeto   \\
		\hline
		Sprint 2 & 22 - 28/set & Refinamento das ideias, adição de um processo na aplicação, divisão de o que cada membro da equipe faria para primeira entrega \\
		\hline
		Sprint 3 & 29/mar - 04/abr  & Finalização da entrega de cada membro para a montagem e treinamento da apresentação \\
		\hline
		Sprint 4 & 5 - 11/abr & Divisão as responsabilidades sobre o desenvolvimento do projeto e início da ambientalização da equipe \\
		\hline
		Sprint 5 & 12 - 18/abr & Definição da arquitetura do projeto. Início da documentação do projeto. Montagem do setup  \\
		\hline
		Sprint 6 & 19 - 25/abr & Continuação da documentação. Preparação do ambiente de integração para a POC \\
		\hline
		Sprint 7 & 26/abril - 02/mai & Desenvolver o desenho da aplicação e montar apresentação. Criação dos wireframes  \\
		\hline
		Sprint 8 & 03 - 09/mai  & Apresentação da POC, início da discução para início do MVP \\
		\hline
		Sprint 9 & 17 - 23/mai  & Criação das páginas de CRUD (back-end e front-end), estilização da aplicação correção dos erros apontados na documentação e finalização da lógica e página de processo  \\
		\hline
		Sprint 10 & 24 - 30/05 & Finalização da documentação do projeto, correções necessárias do MVP \\
		\hline
		Sprint 11 & 31/mai - 04/jun & Finalização do MVP e correções pontuais\\
		\hline
	\end{tabular}
	\par\medskip\ABNTEXfontereduzida\selectfont\textbf
	{Fonte:} Equipe diversaGente (2022) \par\medskip
\end{quadro}
