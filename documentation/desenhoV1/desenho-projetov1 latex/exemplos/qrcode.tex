% ---
\subsection{QR-Code}
% ---
\index{qr-code}
\explicacao{Entendam que faz sentido colocar aqui nesse MODELO uma seção chamada QR-Code pois está sendo explicada a forma de utilização, mas em um documento normal onde o QR-Code é utilizado para apresentar uma URL não faz sentido, já que ele é somente uma ferramenta como um gráfico de pizza}


A utilização de códigos \ac{qr} facilita o acesso de endereços da internet a partir de dispositivos móveis com câmera.
As Figuras \ref{qr-url-1} e \ref{qr-url-2} demonstram dois exemplos de endereços apresentados com essa tecnologia.


Para facilitar a utilização dos códigos \ac{qr}, deve-se tomar cuidado para não deixa-los alinhados na vertical se houverem vários seguidos, pois dificulta a seleção a partir da câmera no dispositivo móvel.

Os endereços também devem ter seu \ac{url} apresentada de forma que mesmo um usuário que esteja fazendo a leitura do documento eletrônico também vai conseguir acessar o endereço indicado. Observe que as figuras de demonstração possuem tanto o código \ac{qr} como o \ac{url}.

Um exemplo para utilização de mais códigos de barra pode ser visto em : \urlmodelo.

Atenção, alguns compiladores podem ter problemas em utilizar a biblioteca \textbf{pstricks} necessária para gerar QR-Codes, no sharelatex em 2017-05 a compilação ocorre perfeitamente utilizando a opção de compilador "XeLatex", ele é mais lento que outras opções.


\begin{figure}[htb]
\caption{\label{qr-url-1}URL para acesso ao documento exemplo}
\begin{pspicture}(25mm,25mm)
\psbarcode{\urlmodelosimples}{eclevel=H width=1.0 height=1.0}{qrcode}
\end{pspicture}
\legend{\urlmodelo}
\fonte{Os Autores.}
\end{figure}


\explicacao{o repositório indicado pela \autoref{qr-url-2} não está sendo atualizado, utilize a versão disponível no overleaf}

% colocando figura qrcode na direita para facilitar o uso da camera deixando cada qrcode em um alinhamento diferente
% se deixar os dois qrcodes um em cima do outro dificulta acessar o desejado
\begin{figure}[htb]
\caption{\label{qr-url-2}Repositório original de classes IFSP \LaTeX}
\begin{flushright}
\begin{pspicture}(25mm,25mm)
\psbarcode{https://github.com/ivanfmartinez/latexlib/tree/master/ifsp}{eclevel=H width=1.0 height=1.0}{qrcode}
\end{pspicture}
\legend{\url{https://github.com/ivanfmartinez/latexlib/tree/master/ifsp}}
\fonte{Os Autores.}
\end{flushright}

\end{figure}

