\section{Normas ABNT}

Esse documento modelo já resolve boa parte da padronização NBR 14.724:2011 \cite{NBR14724:2011} que deve ser seguida e inclusive alguns pontos que não são claros pelo modelo de padronização do \ac{ifsp}.

Leia os documentos do {\abnTeX} e do \ac{ifsp}:
\begin{itemize}
    \item \url{https://www.abntex.net.br/}
    
    \item \acs{faq} : \url{https://github.com/abntex/abntex2/wiki/FAQ}
    
    \item \url{http://mirror.unl.edu/ctan/macros/latex/contrib/abntex2/doc/abntex2.pdf}
    
    \item \waUrl{https://spo.ifsp.edu.br/biblioteca?id=184}
\end{itemize}

No \ac{ifsp} você pode acessar todas as normas \ac{abnt} sem custo, as informações estão disponíveis no endereço \waUrl{https://www.ifsp.edu.br/index.php/outras-noticias/52-reitoria/2329-alunos-e-servidores-do-ifsp-podem-acessar-abnt-via-web.html}.

Apesar de alguns elementos serem opcionais na \ac{abnt} eles foram definidos como obrigatórios (folha de rosto, resumo, lista de siglas, lista de ilustrações, glossário etc), nos trabalhos completos de projetos de informática do \ac{ifsp} campus São Paulo. Documentos menores como propostas de projeto, documento de \ac{poc} não necessitam desses elementos, mas alguns podem ser uteis para ajudar no estudo do {\LaTeX} em preparação para o documento final.

\begin{itemize}
    \item Logotipo da instituição, não é citado na \ac{abnt} nem no manual de normalização do \ac{ifsp}, mas aparece em uma imagem do documento de normalização, foi definido que não deve ser incluído na capa;
    
    \item Nome da instituição que é opcional na capa, deve ser utilizado;
    
\end{itemize}

