% ---
% Conclusão (outro exemplo de capítulo sem numeração e presente no sumário)
% Dependendo do trabalho desenvolvido ele pode ter uma Conclusão ou Considerações finais
% Para trabalhos de disciplina utilizar Considerações Finais
% ---
\chapter{Considerações Finais}
% Exemplo de como adicionar linha adicional no sumário
%\addcontentsline{toc}{chapter}{Considerações Finais}
% Para definir sem número utilizar o asterisco
% Mas se tiver sub seção vai continuar a contagem do capítulo anterior
%\chapter{Considerações finais}



% ---
Além desse documento ser um modelo de como pode ser criado um documento em \LaTeX \space ele também apresenta diversas informações úteis para as disciplinas de projetos de informática do \ac{ifsp} e alguns elementos uteis para as monografias do curso de Pós Graduação em Gestão de \acs{ti} do \ac{ifsp}.

\explicacao{Um trabalho de disciplina não tem \enquote{Conclusão}}

\preencheComTexto


\explicacao{Exemplo de possíveis seções para monografia da pós graduação...}
\section{Resposta à Questão de Pesquisa}
\preencheComTexto

\section{Objetivos Propostos}
\preencheComTexto

\section{Contribuições Acadêmicas e Gerenciais}
\preencheComTexto

\section{Limitações da Pesquisa e Contribuições para Estudo}
\preencheComTexto
