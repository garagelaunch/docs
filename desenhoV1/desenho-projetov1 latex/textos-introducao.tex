
% ----------------------------------------------------------
% Introdução
% ----------------------------------------------------------
\chapter[Metodologia de gestão de projeto e desenvolvimento utilizadas. ]{Metodologia de gestão de projeto e desenvolvimento utilizadas.}

A metodologia que o grupo escolheu para ser usado é o Scrum. Primeiramente pelo fato de ser ágil e também por estarmos em um desenvolvimento de projeto no qual haverá necessidade de programar uma rede social e o Scrum atualmente é amplamente utilizado por equipes de desenvolvedores e também por ser a que o grupo está mais habituado a trabalhar.

Como estamos seguindo a metodologia ágil Scrum, existe uma “hierarquia” e responsabilidades para que o desenvolvimento seja o mais eficiente possível. Para a nossa equipe decidimos optar pelos cargos de Scrum Master, Product Owner e Desenvolvedores.

\begin{itemize}
    \item Scrum Master sendo o responsável para que a estrutura do Scrum seja seguida e realizada. Tira empecilhos de membros da equipe para que o desenvolvimento não seja comprometido, verifica a necessidade de adaptação do Scrum e obter um melhor fluxo de trabalho. O Gustavo Freitas ficou com a função de ser o Scrum Master do time.
    
    \item Tayna França Soza é a Product Owner (PO) da equipe e “representa os interesses de stakeholders (no sentido de agregar valor ao negócio), define funcionalidades do produto e faz a gestão do backlog, priorizando tarefas segundo métodos ágeis, como o Scrum.” (Equipe PM3, 2022)
    
    \item Desenvolvedores são os responsáveis de verificar como que as alterações propostas pelo PO vão ser feitas e se podem ser feitas, integrações, modelagem de banco de dados, programação de código do projeto, definir arquitetura e padronização de commits e documentação do código para que ele tenha uma manutenibilidade alta caso seja necessário manutenções futuras e colocar a aplicação em produção sem nenhum problema futura. Os responsáveis por essas funções são: Gabriel Ruiz, Grazielli Berti, João Bispo, Vinicius Soares e Viviane Queiroz.
    
\end{itemize}

