\subsection{Tabelas e Quadros}
\label{tabelas-e-quadros}
Quadros e Tabelas são informações tabulares, mas Tabelas tem como objetivo apresentar números. A ‘norma’ 14724 \cite[3.32]{NBR14724:2011} define a Tabela como sendo uma \enquote{forma não discursiva de apresentar informações das quais o dado numérico se destaca como informação central} e que devem seguir padronização do \ac{ibge}  \cite[5.9]{NBR14724:2011}. O \ac{ibge} padronizou a apresentações de dados tabulares em 1993 \cite{tabular-ibge}.

Informações adicionais sobre o de tabelas no {\LaTeX} podem ser obtidas em  \url{https://en.wikibooks.org/wiki/LaTeX/Tables}.

Antes de utilizar \index{longtable}\textbf{longtable} procure reorganizar o seu layout ou quebrar manualmente em múltiplos quadros / tabelas, pois isso ainda facilita a compreensão pelo leitor.

% https://biblioteca.ibge.gov.br/visualizacao/livros/liv23907.pdf

\index{quadros}O \autoref{quadro-exemplo} é um exemplo de dados tabulares gerados em 
\LaTeX.



\begin{quadro}[htb]
\centering
\ABNTEXfontereduzida
\caption[Níveis de investigação]{Níveis de investigação}
\label{quadro-exemplo}
\begin{tabular}{|p{2.6cm}|p{6.0cm}|p{2.25cm}|p{3.40cm}|}
  \hline
   \thead{Nível de\\Investigação} & \thead{Insumos}  & \thead{Sistemas de\\ Investigação}  & \thead{Produtos}  \\
    \hline
    Meta-nível & Filosofia\index{filosofia} da Ciência  & Epistemologia &
    Paradigma  \\
    \hline
    Nível do objeto & Paradigmas do metanível e evidências do nível inferior &
    Ciência  & Teorias e modelos \\
    \hline
    Nível inferior & Modelos e métodos do nível do objeto e problemas do nível inferior & Prática & Solução de problemas  \\
   \hline
\end{tabular}
\fonte{O Autor.}
\end{quadro}



\index{tabelas}Já a \autoref{tab-exemplo} foi criada conforme o padrão \citeonline{tabular-ibge} requerido pelas normas da \ac{abnt} para documentos técnicos e acadêmicos. Observe que não existem bordas laterais e nem linhas separadoras em uma Tabela e as colunas numéricas tem alinhamento à direita. 

\begin{table}[htb]
\centering
\caption{Métricas de desenvolvimento}
\label{tab-exemplo}
\begin{tabular}{p{2.6cm}rrr}
    \hline
   \thead{Item} & \thead{Janeiro}  & \thead{Fevereiro}  & \thead{Março}  \\
    \hline
    Classes & 2  & 10 & 20  \\
    Linhas & 100  & 250 & 543 \\
    \hline
\end{tabular}
\fonte{Os autores.}
\end{table}

\def\equationautorefname~#1\null{%
  Equação~(#1)\null
}


Para facilitar a criação de tabelas e quadros existem algumas ferramentas como o Tables Generator \url{http://www.tablesgenerator.com/latex_tables} que permite a criação de forma visual gerando o código \LaTeX\ correspondente. E o site \url{https://www.latex-tables.com/} permite converter planilhas em código \LaTeX.


\index{equação}\index{Pitágoras}A \autoref{eq-pythagoras} demonstra que também é possível escrever equações diretamente em \LaTeX

\begin{equation}\label{eq-pythagoras}
a^2+b^2=c^2\,.
\end{equation}





